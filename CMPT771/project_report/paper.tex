%File: formatting-instruction.tex
\documentclass[letterpaper]{article}
\usepackage{aaai}
\usepackage{times}
\usepackage{helvet}
\usepackage{courier}
\frenchspacing
\pdfinfo{
/Title (Formatting Instructions for Authors Using LaTeX)
/Subject (AAAI Publications)
/Author (AAAI Press)}
\setcounter{secnumdepth}{1}  
 \begin{document}
% The file aaai.sty is the style file for AAAI Press 
% proceedings, working notes, and technical reports.
%
\title{Hop Probability Based Partitioning of Online Social Networks}
\author{Vivek Mahajan\\
vmahajan@sfu.ca\\
}
\maketitle
\begin{abstract}
\begin{quote}
Online social networking has  become a phenomenon in the last few years. The
growing usage of social networks has put the service providers under a lot of
pressure to efficiently process the users queries.  In social networks like
facebook, most of the queries access the immediate neighbors of the current node
and neighbors of its neighbors.  Existing graph partitioning approaches focus
on reducing the number of inter server links, neglecting the query patterns.
Our algorithm focuses on partitioning the graph so as to minimize the number of
servers to be accessed in order to process a query. Our results show that the
proposed algorithm gives useful partitioning of social networks, thereby
reducing inter server accesses.


\end{quote}
\end{abstract}

\section{Introduction}
    In the last few years, the online social networks(OSNs) have grown in huge
proportion with respect to the number of users as well as its utility. The
amount of usage of OSNs exceeds e-mail usage nowadays, which is remarkable.
OSNs are the most popular sites on the Internet. A
study\footnote{http://www.news.com.au/technology/story/0,28348,25170441-5014239,00.html}
reported that OSNs account for 10\% of all the Internet time worldwide. The
growth can be appreciated by the fact that there was 1382\% increase in the
user base of twitter in one month[Feb-Mar 09]
\footnote{http://blog.nielsen.com/nielsenwire/onlinemobile/twitters-tweet-smell-of-success}.
There can be variations of OSNs in terms of usability e.g On facebook, the users
interact with each other by posting on the walls or sharing photographs, while on
SLASHDOT, the users interact via blogs. LinkedIn and Google+ are other examples
of OSNs.

    The phenomenal growth in the size of OSNs has given rise to scalability
issues that can be solved by distributing online social network graph among
different servers. However finding the basis of partitioning  is a major issue. That
is, the graph partitioning could be based on structural, geo-political or
traffic properties, which are discussed in detail in \cite{arxviv} for twitter
and orkut. In this paper, we introduce a new partitioning criterion called
\texttt{Hop probability based partitioning} which takes into consideration the typical
nature of social network queries.

\subsection{Hop probability Based Partitioning}

Social network queries typically visit their one hop and two hop neighbors. The
small world phenomenon links any two users on a social network within a distance
of six hops. Hence, while partitioning a social network, it would be beneficial
to maintain the one-hop and two-hop neighbors of a frequented query node on the
same server\footnote{through the rest of the paper the server and partition
terms are used interchangeably} as itself.  

For example, \cite{flickr} analyzed large-scale traces of information
dissemination in the Flickr social network. Flickr allows users to create two
types of links: links to favorite photos (called favorites in Flickr) and links
to other users (called contacts in Flickr). Users may \texttt{favorite-mark} a
photo and share interesting photos with others. \cite{flickr} refers to users
who include a photo in their favorite photos list as fans of that photo. It was
observed that for less popular photos, 91\% of all fans are within two hops of
the uploaders. Further, for even for top popular photos, 81\% of all fans are
within two hops of uploaders. \cite{OHM} mentions that most information is
found one-hop away from the query node. According to a study\cite{4}, the
maximum distance between two persons is six hops. Hence, it is a reasonable
assumption that a user interacts primarily within his 2-3 hop radius and at
most his six hop radius.

    In this project, we propose a novel hop probability based objective function
and  an algorithm to find the clusters which optimizes that. Given a social
network graph $G$, our algorithm partitions the graph while meeting the
following constraints:
\begin{itemize}
\item For any node in a partition most of its two-hop neighbors lie within the same partition.
\item All partitions are approximately equally sized.
\end{itemize} 

The paper is organized as follows. The related works is shown in section 2, the
objective function in explained in section 3, We describe our algorithm  in
section 4 and  experimental results are shown in section 5.  Finally, we
conclude the paper in section 6.

\section{Related Works}
  The study of graph partitioning algorithms based on different properties of
social networks has been proposed by \cite{arxviv}. In the recent past, graph
partitioning for dealing with scalability issues and handling large datasets
has been proposed by \cite{5,6}.

    With the increased usage of social networks, the scientists have become
more interested in analyzing the properties of social networks.  Sociologists
showed that the average path distance between two Americans is 6 hops \cite{4}
and \cite{8} shows the small world effect analysis.  Study of a online social
network  finds the small world behavior and local clustering property
\cite{9}.  The small world property for offline social networks has been
discussed in \cite{9,10,11} and it was verified for the online social network
\cite{12,13}. Analysis of different structural and scale free properties has
also been extensively analyzed \cite{12}.



    The traditional graph partitioning algorithms\cite{16,17,18} partition the
graph into roughly equal sized partition  and focusses on minimizing the edge
cuts.  Further, the sociological approaches like hierarchical clustering based
partitioning and natural division based partitioning approaches are popularity
known\cite{19}. Another genre of graph partitioning algorithms that focus on
producing quality partitions in less time are multilevel algorithms, which
include METIS\cite{20} and its modification for power law graphs\cite{21}.
There are other partitioning algorithms like Kernighan-Lin\cite{22} and it's
linear time heuristics\cite{23} can also be used as refinement algorithms.


    In this paper, we  compare our results with METIS\cite{20}. METIS focuses
on minimizing the cuts between the partitions as well as into roughly equal sized
partitions. The METIS algorithm is divided into three phases termed as
coarsening, partitioning and uncoarsening/refinement.  The main motivation
behind the algorithm is to generate an approximate graph of the original graph,
partitioning it and then retracting the coarsening phase while doing the
refinement between each iteration.  Our algorithm also follows a hierarchical
clustering approach where repeated merging of clusters is done for some
iteration and then the graph is randomly partitioned with imposed constraints of
having nearly equal partitions. Finally, the refinement algorithm is applied on
the clusters thus obtained. We differ from METIS, in terms of our partitioning
criterion, our novel way of clustering, and  the modified form Kernighan-Lin 
algorithm which  refines the generated partitions. In coarsening phase, our
algorithm tries to collapse the nodes into clusters so as to minimize the
evaluation criteria, while in METIS it doesn't take that into account. The refinement
phase also focuses on minimizing the evaluation criteria. 
\section{Hop probability Based Objective Function}

   To calculate our objective function, we assume that there is an equal chance
of each user of an OSN to come online. However, we can easily modify our
objective function if we know the probability of each user to come online.

 Given partitions of a graph, our evaluation criteria gives a good
quantitative measure of the quality of partitions in terms of fraction of one and two hops
lying outside the partition and should also give importance to the past
interaction between the users of OSNs. Essentially, our objective function
calculates the average probability of one and two hops from a node to lie in different
partition. 

Formally, let $X$ = $\{x_1,x_2,x_3,\cdot\cdot\cdot,x_n\}$ be the nodes in an OSN, $x_i$ = $i^{th}\ node$, 
$Y_i=\{y_1, y_2, y_3,\cdot\cdot\cdot,y_j\}$ be the $j$ neighbors of $x_i$. Let $V_i=\{v_1, v_2, v_3, \cdot\cdot\cdot,
 v_j\}$ be the number of interactions between $x_i$ and its neighbors. And $|V_i| = \sum_{k\in j} v_k$.
 Let $Y^\prime_i$ be 
the subset of $Y_i$ which lie in partition other than $x_i$'s. Let $Y^\prime_i = \{y_1^\prime, y_2^\prime, \cdot\cdot, y_k^\prime\}$. 
Let the interaction between $x_i$ and the users in this subset be  $V^\prime_i = \{v_1^\prime, v_2^\prime, \cdot\cdot, v_k^\prime\}$.
$|V_i^\prime| = \sum_{p\in k} v_p^\prime$.

        Let $P_{x_i}^{1-hop}$ be the probability of one-hop neighbor of $x_i$ to lie in separate partition, and $P_{x_i}^{2-hop}$ be probability of two hops from $x_i^{th}$ node to lie in the separate partition.\\
        Clearly, $P_{x_i}^{1-hop}= \frac{|V_i^\prime|}{|V_i|}$ \\
        $P_{x_i}^{2-hop}=\frac{1}{Number\ of\ neighbors\ of\ x_i} * \displaystyle\sum_{k=1}^j P_{x_i y_k}$, \\
        where,\\
        $P_{x_i y_k}$ = $\frac{v_k}{|V_i|}$* $P_{x_{k}}^{1-hop}$ =  Probability of two hop from  $x_{i}$ to lie  in a  partition other than  its own through $y_k$, if $x_i$ and $y_{k}$ are in the same partition.\\
        $P_{x_i y_k}$ =  1 , if $x_i$ and $y_k$ are in the separate partition, as we assume if the one hop is on a different partition, it is almost
certain that two hop lie in the same partition.\\
        $Probability\ Index=\frac{1}{Total\ number\ of\ nodes}*\displaystyle\sum_{i=1}^nP_{x_i}^{2-hop}$, $n$ is the total number of nodes.\\


        This probability index gives us an overall probability of a two hop
query from a node to lie in the different partition. To the best of our
knowledge this kind of evaluation of evaluation criteria has never been
published before. This is quite novel in the sense it takes into account the
user access patterns.  In the following section we will propose an algorithm
which optimizes this objective function.

\section{Our Algorithm}
\begin{table*}[htb]
\centering
\begin{tabular}{|l|}%putting \begin{tabular}{|l|} gives table with side-lines.
\hline
\textbf{Our Algorithm}\\
\hline
\textbf{Input: }Graph, No of Partitions, No. of iterations\\
\hline
\textbf{Phase 1 - Graph Coarsening }\\
1. Node weight calculation$()$ - Calculate the weights for each node according \\
~~~ to the algorithm discussed in section 3.2\\
2. Clustering$()$ - Forming clusters according to the sorted order\\
~~~ of the weights.\\
3. Graph formation from clusters$()$ - Consider a cluster as a node for the\\
~~~ graph and collapse the edges to form a graph\\
3. Repeat step 1-3 for the desired number of iterations.\\


\textbf{Phase 2 - Partitioning}\\
 Randomly allocate nodes into the desired number of clusters. We can use any\\
traditional approach towards graph partitioning, in our algorithm  we are using\\
a randomized one\\

\textbf{Phase 2 - Refining}\\
 Refine the partitions using Kernighan-Lin algorithm or the modified one.\\
\hline
\end{tabular}
%\vskip -0.1in
\caption{Our Algorithm}
\label{alg}
\end{table*}

In this section, we introduce our graph partitioning algorithm. A social
network  partitioning algorithm is expected to reduce inter server accesses,
requiring friends and the friends of friends of a query node to be present in
the same server as itself.   Also, in order to equally distribute the workload
across servers , it is essential that the partitions size should be
approximately equal.


 Our algorithm attempts to fulfil the above requirements
while attempting to maximize the number of nodes for which a majority of its
one-hop neighbors and with next priority its 2-hop neighbors are included
within the same server.  Our algorithm is divided into three
phases, namely the graph coarsening phase, partitioning phase and the
refinement phase. The graph coarsening phase reduces the size of the original
graph by some factor by clustering the nodes \emph{intelligently}.  A weight
function is used in order to decide which nodes should be clustered  during the
coarsening phase. In the partitioning phase, we randomly generate partitions of
the coarsened graph. Due to time constraint we decided to randomly generate the
partitions. Alternatively, we can use traditional graph partitioning methods.
In the next phase, which we call refinement phase we reduce the inter partition
edges by using a Kernighan-Lin algorithm. We next give definitions of commonly
used terms followed by a detailed description of each of the phases.

\subsection{Definitions}
\textbf {\it{Social Network Graph :} }The users in the social network form the
nodes of the graph while the relationships between the users form the edges of
the social network graph.\\ \textbf{\it{k-hop neighbor :}} A node is said to be
a k-hop neighbor of a query node if there are  $K$-edges separating the node
from the query node. A one hop neighbor can also be a two hop neighbor and so
on in case of triangles present in the graph

\subsection{Coarsening Phase}

The aim of this phase is to produce an approximate partitioning of the original
graph. Our approach is a hierarchical one.  We start with the original graph
and then we try to reduce the graph size by \emph{intelligently} clustering
nodes at each level. After clustering nodes, we form a new graph with clusters
from the previous level treated as nodes in the current level. After repeating
the steps for some iterations, we apply a traditional graph partitioning
algorithm to \emph{latest} graph. And then we do the refining on the produced
clusters to get the desired output. In the following sections, we will describe
our \emph{intelligent} clustering approach at each level and then we will
continue our discussion with the explanation of the refining algorithm.
\begin{itemize} % 1

\item \textbf{Graph Definition:}

Graph  $G$=$(V,E)$, where $V$ = $\{v_1,v_2,v_3,\cdot\cdot\cdot,v_n\}$ ,$v_i$ represents a cluster of nodes. 
In the first level, $v_i$ consists of only single node and $E$ is the edge set which consists of $e_{ij}$ i.e the number of interactions between the $v_i$ and $v_j$ i.e for any $v_i$ and $v_j$,if there are $k$ edges between 
the two then $e_{ij}$ will be equal to $k$, for the first level $e_{ij}=number\ of\ interactions\ between\ v_i\ and\ v_j$.\footnote{number of interactions might the average number of times user $v_i$ and $v_j$ have interacted
with each other} 

\end{itemize} % 1

\begin{itemize} % 1

\item \textbf{Node Weight Calculation:} In this project, we adopt a probability
based weight function.  Nodes of higher weight values are later picked as seeds
during the process of clustering. 

\begin{itemize} %2

\item \textbf{Probability based weight function:}  
In section 3, we described the $P_{x_i}^{2-hop}$ as the probability of two hops
from $x_i^{th}$ to lie in separate partition. To calculate the weight for each
node $x_i$, we calculate $P_{x_i}^{2-hop}$ assuming the nodes and its neighbors
are in one cluster and rest of the nodes are in separate cluster. Higher the
weight, more are the chances that even if we have the node and its neighbors in
one cluster, still there is high chance of two hops from node $x_i$ lying in
separate cluster. So we start by clustering $\emph{low weight}$ nodes and we
continue in ascending order.
 


\end{itemize} % 2



\item \textbf{Clustering Algorithm:} The algorithm behaves like a hierarchical
clustering algorithm where at each level clusters are formed by clustering the
clusters of the previous level. The clustering algorithm varies based on the
weight picking function used as described below.



For a given level $i$, the nodes of the graph at level $i$ are processed in
decreasing order of weight. Initially the input graph forms the input of the
clustering algorithm for level one. For each level $i >1$, a new graph is
generated which is fed as input to the clustering algorithm.

\begin{itemize} % 2

\item \textbf{Probability based clustering:} Each selected node assigns all its immediate neighbors to a new cluster. 
A neighbor that has already been assigned to another cluster in the same iteration of level is ignored. 
The process is repeated until every node is assigned some cluster number.



\end{itemize} % 2

A predefined cluster size is set as:\\

$Cluster$ $Size$ = ($Number$ $of$ $Nodes$ / $Number$ $of$ $partitions$) $+$/$-$
$deviation$ $value$.\\

where $deviation$ $value$= $x\%$ of ($Number$ $of$ $Nodes$ / $Number$ $of$
$partitions$) for some user defined value of $x$.\\ 


The user defined value $x$ in the equation above is referred to as the "limit
value" through the rest of the paper.

The size of any cluster is required to be less than the predefined $Cluster$
$Size$. If no node can be selected as a seed node as on selecting it would be
giving rise to a cluster that exceeds the $cluster$ $size$, then some
heuristics applied to cluster the remaining nodes in a graph.  

A cluster that cannot accommodate further nodes is marked and the new graph
formed ignores the node representing this cluster for further clustering. 



\item \textbf{Graph Formation from Clusters:} For the first level before the
algorithm starts, each node of the input graph is represented as a cluster in
itself. After the first level of clustering at level one, a new graph is
generated to act as the input node for level two on which the four process are
repeated. A new graph $G_i=(V_i, E_i)$ is formed for the clustering at level $i
> 1$ after the clustering at level $i-1$. Each cluster formed at a given level
$i-1$ is denoted as a new node. Hence the number of clusters at level $i-1$
form the number of nodes at level $i$. The intra cluster edges form the edge
set $E_i$ of the graph $G_i$.




\end{itemize} % 1

\subsection{Partitioning Phase}
After coarsening the graph, we can apply any traditional graph partitioning
approaches to partition it. In our case, we are randomly select nodes and put
them in partitions. While partitioning we also make sure that the size of does
not differ by large quantity i.e they should be of nearly equal size.
%\end{itemize} % 1

 
\subsection{Refinement Phase}
When the required number of partitions are formed, it gives us an approximate
partition of the graph. To refine the partitions, we use a modified form of
Kernighan-Lin algorithm\cite{22}.


    The Kernighan-Lin algorithm starts with initial Bi-partitions of the graph
and then iteratively optimizes it using a greedy approach. The Kernighan-Lin
algorithm is divided into two phases, in the first phase, we consider a pair of
nodes, with each nodes from different partitions. Now, a gain value $\delta$$Q$
is calculated for all possible pairs, which is the change in the number of
edges when the nodes in the pair are swapped. The pair with the maximum  gain
is marked for swapping so that they are never marked again in this phase. This
continues until all the pairs are marked. In the next phase, the algorithm
examine the sequence of the swapped pairs and the sequence which results into
the maximum gain is selected and the actual swapping takes place. This results
into a locally optimized graph and the two phases are repeated till there is no
further gain in the edges between the partitions.


    We modify one of the phases of  Kernighan-Lin algorithm for our criterion.
To be precise, we change the definition of $\delta$$Q$ for us. In the
Kernighan-Lin algorithm, $\delta$$Q$ is the change in the edges cuts between
the partitions when the nodes are swapped. Let $a$, $b$ be two nodes that lie
in different partitions. Let $I_a$ be the internal cost for $a$, i.e the total
number of interactions between $a$ and its friends which lie in the same
partition and  $E_a$ be the external cost for the node $a$, i.e the total
number of interaction between $a$ and its friends which lie in the separate
partition. $C_{a,b}$ is the average number of interactions
between user $a$ and $b$. Similarily $E_b$ and $I_b$ are the external cost and internal cost
for the node $b$. Now let $D_a$ = $E_a$ - $I_a$ and $D_b$ = $E_b$
- $I_b$ , i.e the difference of external and internal costs. If $a$ and $b$ are
  interchanged then the gain i.e $\delta$$Q$ will be $D_a + D_b -2C_{a,b}$. We
make changes in the internal, external and $C_{a,b}$ to fit our purpose of
decreasing the probability of the two hops from one node lying in the different
partitions. We define internal cost $I_{ap}$ as the probability of an edge of
node $E_{ap}$ lying in the same partition as of $a$ ,external cost as
probability of an edge of node $a$ lying in a separate partition. Let $T_a$ be
$Total\ number\ of\ interactions\ done\ by\ user\ a$ and $T_b$ be $Total\
number\ of\ interactions\ done\ by\ user\ b$. And $I_{ab}$ be the number of
interactions between $a$ and $b$. Hence,\\
\begin{displaymath}I_{ap}=\frac{I_a}{T_a},\\\end{displaymath}
\\\begin{displaymath}E_{ap}=\frac{E_a}{T_a}\\\end{displaymath} and\\
\begin{displaymath}2C_{ap,bp}=\frac{I_{ab}}{T_a}+\frac{I_{ab}}{T_b}\\\end{displaymath}


    As the original Kernighan-Lin algorithm focuses on reducing the edges
between the partitions, our modified one reduces the probability of one hop of
the nodes to lie in the separate partition.  The rest of the Kernighan-Lin
algorithm works in the original way. Experiments shows that it not only reduces
the probability of one hop to lie outside the partitions but also decreases the
edge cuts. The same approach could be applied to the linear heuristics from the
feduccia and Mathews algorithm\cite{23} which can make it more efficient in
terms of time taken to refine the partitions.
 
\begin{table*}[htb]
\centering
\caption{Datasets - statistics}
\begin{tabular}{|c|c|c|c|}
\hline
 Datasets& Nodes & Edges \\
\hline
Facebook& 63731 & 68885  \\
\hline
SOC-Epinions & 75879 & 508837   \\
\hline
SLASHDOT& 77360 & 905468  \\
\hline
\end{tabular}

\label{fig:twopartcuts}
\end{table*}
\begin{table*}[htb]
\centering
\caption{Two Partitions - Evaluation Indexes}
\begin{tabular}{|c|c|c|c|c|c|c|}
\hline
 &\multicolumn{2}{c|}{Our Algorithm}&\multicolumn{2}{c|}{METIS}\\
\hline
%\cline{2-5}
 Datasets&Hop Probability score &Edge Cuts&Hop Probability Score&Edge Cuts\\
\hline
Facebook& 0.042590 & 87642  & 0.031047 & 68315\\
\hline
SOC-Epinions & 0.192315 & 62544  & 0.163753 & 55104\\
\hline
SLASHDOT& 0.398257 & 113356  & 0.438645 & 81465 \\
\hline
\end{tabular}
\label{fig:twopartindex}
\end{table*}





\section{Experiments and Results}
We have picked three popular and reasonably sized datasets to run our experiments on
\begin{itemize}
\item \textbf{Facebook Social Network}: Facebook is a true social networking
site on which users interact by writing on the wall-posts and sharing lots of
media content. The dataset was collected between December 29th, 2008 and
January 3rd, 2009, and between January 20th 2009 and January 22th 2009 from the
Facebook New Orleans network. The original dataset is time stamped, mentioning
when were the links made.  For our purposes we used it as an undirected graph
giving no consideration to time stamp attribute.  We also have list of all the
wall posts for Facebook New Orleans network. The dataset was anonymized and we
only have information about which user posted on whose wall at what time. No
posts content is provided. But this information was good enough to find the
interactions between the users which forms the weights of the edges between the
users.  So we build a graph in which nodes are represented by users and there's
and edge between two nodes if those two users are friends and weights are the
average number of interactions between the two users i.e average number of
times they posted on each other's wall + 1. 1 is added as there were cases when
two users were friends but they didn't interact with each other. So we decided
on having edge weight as 1 by default. As at-least they interacted once when
they accepted the friends request. The dataset was crawled by \cite{26}. We have chosen
this dataset as it is one of the few Facebook datasets provided on the web, which also has
the wall-posts information during that period. That gives us a close approximate of
the users interaction amongst each other.

\item \textbf{Epinions Social Network}: This is who-trust-whom online social
network of a general consumer review site Epinions.com. Members of the site can
decide whether to "trust" each other.  All the trust relationships interact and
for the web of trust which is then combined with review rating to determine
which reviews are shown to the user. We construct a graph in which the users
are represented by nodes and there is an edge between a user if they "trust"
each other. Since we didn't have any user query access pattern information for
this graph, we gave each edge a weight of 1. The
dataset\footnote{http://snap.stanford.edu/data/soc-Epinions1.html} was first
used in \cite{epi2003}.

\item \textbf{Slashdot}: Slashdot is a technology related news website known
for its specific user community.  The website features user submitted and
editor evaluated current primarily technology oriented news.  In 2002 Slashdot
introduced the Slashdot Zoo feature which allows users to tag each other as
neighbors or foes.  The network contains friend/foe links between the users of
Slashdot. Again, the users are nodes and a link between them indicates whether
they are friend or foe. And edge weight is set to 1 as we don't have any access
pattern information. The
dataset\footnote{http://snap.stanford.edu/data/soc-Slashdot0811.html} was first
introduced in \cite{25}.

\end{itemize}
We compare the hop probability based partitioning approach with METIS. We are
considering only Bi-partitions in our results. The input graph to the two
algorithms are same except the facebook graph. For our algorithm, the input
Facebook graph is with edge weights based on  interactions between users, while
input Facebook graph for METIS is with all the edge weights set to 1.
We also have a reasonable constraint that the partitions should be of
roughly equal size. We compare the two algorithms on edges cuts and hop
probability based evaluation criteria. 

As we can see the preliminary results
are quite close to METIS in terms of edge cuts as well the novel criteria. We
think that some more experimentation with the weight function, partitioning
algorithm(here we are doing random partitioning) and the refinement algorithm,
might result in better results. Given the time span for the course project,
we thought of running our algorithm on relatively small graphs. We wanted to check the performance
in terms of edge cuts and the probability based evaluation criteron. We feel that
since the first iteration of the algorithm gave us quite close results, we beleive further improvement
over them will beat the more refined $METIS$ algorithm which took few years to develop. 


\section{Conclusion and Future Work}
In this paper, our focus was to partition an Online Social Network graph such
that the probability of query(one-hop/two-hop) generated from a user lying in
different partition is reduced. We proposed a novel criteria for evaluating the
partitions and presented a hierarchical algorithm optimizing that criteria. Our
preliminary results looks promising. We plan to further improve it by 
analyzing the graph formation at each level and also by experimenting with the
weight function as well as the partitioning function. Refinement algorithm can
be changed as well. As we don't know whether the modified Kernighan Lin
algorithm will surely converge. We also want to compare the algorithm 
with variety of graph partitioning algorithm other than METIS and also design 
an algorithm for generating $k$ partitions. I plan to work on this a little more and 
 eventually submit it as a workshop paper in ICWSM'13.
\bibliographystyle{aaai} \bibliography{xbib}
\end{document}
