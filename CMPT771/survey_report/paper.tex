%File: formatting-instruction.tex
\documentclass[letterpaper]{article}
\usepackage{aaai}
\usepackage{times}
\usepackage{helvet}
\usepackage{courier}
\frenchspacing
%\pdfinfo{
%/Title (Formatting Instructions for Authors Using LaTeX)
%/Subject (AAAI Publications)
%/Author (AAAI Press)}
\setcounter{secnumdepth}{1}  
 \begin{document}
% The file aaai.sty is the style file for AAAI Press 
% proceedings, working notes, and technical reports.
%
\title{Finding communities in graphs}
\author{Vivek Mahajan\\
vmahajan@sfu.ca\\
}
\maketitle
\begin{abstract}
\begin{quote}
Detecting clusters or communities in large real-world graphs such as large social or information networks is a problem of considerable interest. In general, it is the 
organization of vertices in clusters, with many edges joining vertices of the same cluster and comparatively few edges joining vertices of different clusters. 
In practice, one typically chooses an objective function that captures the intution as given by the definition above, and then one applies approximation algorithms
or heuristics to extract sets of nodes which appears to be "good" with respect to that objective function.

In this survey, we will start with the applications of clustering in various domains. We will describle various objective functions and will also discuss some of the
most cited community finding papers. 

\end{quote}
\end{abstract}

\section{Introduction}

\section{Objective Functions}
If you have clusters of a given graph as an output of a clustering algorithm, we need to have some quantitative measure of how "good" they are. We call them objective
function which the algorithms try to optimize. The intution behind most of the objective functions is that a good cluster has many edges internally and few edges 
pointing outside. In this section, we look at the other objective functions that are based on this intution.


In general, the two criterias contributes heavily to the formulation of objective function. The first is the number of edges between the members of the clusters, and 
second is the number of edges between the members of the cluster and remainder of the network. So, based on whether an objective function is using either one or both 
criteria in its formulation, we can divide the objective functions into two categories\cite{leskovec2010}. The first group, that we refer to as Multi-criterion scores, combines both 
criteria into a single objective function; while the second group of objective functions uses only  a single of the two criteria. We will give a quick overview of the
various definitions of the objective functions which falls under the two categories. 

\subsection{Multi-criterion scores}
Let $G(V,E)$ be an undirected graph with $n=|V|$ and $m=|E|$ edges. Let S be the set of nodes in the cluster, where $n_S$ is the number of nodes in S,
$n_S=|S|$; $m_S$ the number of edges in $S$,  $m_S=|\{(u,v):u\in S, v\notin S\}|;$ and $d(u)$ is the degree of node $u$.

We consider the following metrics $f(S)$ that capture the notion of a quality of the cluster. Lower value of score $f(S)$ signifies a more community-like set of 
nodes.
\begin{itemize}
\item \textbf{Expansion}:  
\item \textbf{Internal density}:  
\item \textbf{Cut Ratio}:  
\item \textbf{Normalized Cut}:  
\item \textbf{Maximum-ODF(Out Degree Fraction)}  
\item \textbf{Average-ODF}:  
\item \textbf{Flake-ODF}: 
\end{itemize}

\subsection{Single-criterion scores}
\begin{itemize}
\item \textbf{Modularity}:
\item \textbf{Modularity ratio}:
\item \textbf{Volume}:
\item \textbf{Edges cut}:
\end{itemize}

\section{Algorithms}

\cite{test123}
\bibliographystyle{aaai} \bibliography{bibfile}
\end{document}
